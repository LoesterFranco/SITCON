%%%%%%%%%%%%%%%%%%%%%%%%%%%%%%%%%%%%%%%%%
% Beamer Presentation
% LaTeX Template
% Version 1.0 (10/11/12)
%
% This template has been downloaded from:
% http://www.LaTeXTemplates.com
%
% License:
% CC BY-NC-SA 3.0 (http://creativecommons.org/licenses/by-nc-sa/3.0/)
%
%%%%%%%%%%%%%%%%%%%%%%%%%%%%%%%%%%%%%%%%%

%----------------------------------------------------------------------------------------
%	PACKAGES AND THEMES
%----------------------------------------------------------------------------------------

\documentclass{beamer}
\usepackage[utf8]{inputenc}

\mode<presentation> {

% The Beamer class comes with a number of default slide themes
% which change the colors and layouts of slides. Below this is a list
% of all the themes, uncomment each in turn to see what they look like.

%\usetheme{default}
%\usetheme{AnnArbor}
%\usetheme{Antibes}
%\usetheme{Bergen}
%\usetheme{Berkeley}
%\usetheme{Berlin}
%\usetheme{Boadilla}
%\usetheme{CambridgeUS}
%\usetheme{Copenhagen}
%\usetheme{Darmstadt}
%\usetheme{Dresden}
%\usetheme{Frankfurt}
%\usetheme{Goettingen}
%\usetheme{Hannover}
%\usetheme{Ilmenau}
%\usetheme{JuanLesPins}
%\usetheme{Luebeck}
\usetheme{Madrid}
%\usetheme{Malmoe}
%\usetheme{Marburg}
%\usetheme{Montpellier}
%\usetheme{PaloAlto}
%\usetheme{Pittsburgh}
%\usetheme{Rochester}
%\usetheme{Singapore}
%\usetheme{Szeged}
%\usetheme{Warsaw}

% As well as themes, the Beamer class has a number of color themes
% for any slide theme. Uncomment each of these in turn to see how it
% changes the colors of your current slide theme.

%\usecolortheme{albatross}
%\usecolortheme{beaver}
%\usecolortheme{beetle}
%\usecolortheme{crane}
%\usecolortheme{dolphin}
%\usecolortheme{dove}
%\usecolortheme{fly}
%\usecolortheme{lily}
%\usecolortheme{orchid}
%\usecolortheme{rose}
%\usecolortheme{seagull}
%\usecolortheme{seahorse}
%\usecolortheme{whale}
%\usecolortheme{wolverine}

%\setbeamertemplate{footline} % To remove the footer line in all slides uncomment this line
%\setbeamertemplate{footline}[page number] % To replace the footer line in all slides with a simple slide count uncomment this line

%\setbeamertemplate{navigation symbols}{} % To remove the navigation symbols from the bottom of all slides uncomment this line
}

\usepackage{graphicx} % Allows including images
\usepackage{booktabs} % Allows the use of \toprule, \midrule and \bottomrule in tables

%----------------------------------------------------------------------------------------
%	TITLE PAGE
%----------------------------------------------------------------------------------------

\title[Libre Silicon]{Libre Silicon} % The short title appears at the bottom of every slide, the full title is only on the title page

\author{Hagen SANKOWSKI} % Your name
\institute[Chipforge] % Your institution as it will appear on the bottom of every slide, may be shorthand to save space
{
Chipforge\\ % Your institution for the title page
\medskip
\textit{hsank@nospam.chipforge.org} % Your email address
}
\date{\today} % Date, can be changed to a custom date

\begin{document}

\begin{frame}
\titlepage % Print the title page as the first slide
\end{frame}

%----------------------------------------------------------------------------------------
%	PRESENTATION SLIDES
%----------------------------------------------------------------------------------------

%------------------------------------------------
\begin{frame}
Current Situation
\end{frame}
%------------------------------------------------

%------------------------------------------------
\begin{frame}
What is bad..
\end{frame}
%------------------------------------------------

\begin{frame}
\frametitle{Image you like to manufacture your own Chip.}
\begin{itemize}
\item You're going to a Foundry,
\item signing at least 3 NDAs (Non-disclosure Agreements), one for the Process Kit, one for the Standard Cell Libary and one for Purchase details,
\item invest a lot of money for the Layout development and the Mask Set,
\item and have some reasons to change the Foundry Service..
\end{itemize}
\end{frame}

%------------------------------------------------

\begin{frame}
You're f*cked
\end{frame}
%------------------------------------------------

\begin{frame}
\frametitle{Reasons are}
\begin{itemize}
\item the technology is completely different,
\item the Standard Cells are mostly different,
\item the mask the does not leave the foundry,
\item and even do not match another technology in another foundry.
\item Well, you've burned the costs for layout and mask set.
\end{itemize}
\end{frame}

%------------------------------------------------
\begin{frame}
What to do??
\end{frame}
%------------------------------------------------

\begin{frame}
\frametitle{Make your self independend}
\begin{itemize}
\item design a open and free process.
\item You can help if you like :-)
\end{itemize}
\end{frame}

%------------------------------------------------
\begin{frame}
What happens so far?
\end{frame}
%------------------------------------------------

\begin{frame}
\frametitle{2017}
\begin{itemize}
\item David Lanzendörfer opens a possibility to rent a Clean Room at Hong Kong University of Science and Technology,
\item got some foundations,
\item gave a Lightning Talk in Leipzig at the 34. Chaos Communication Congress.
\end{itemize}
\end{frame}


\begin{frame}
\frametitle{2018}
\begin{itemize}
\item We developed the first Version of our 1um Libre Silicon process.
\item We are working on the Standard Cell Library.
\item We already hold a Tool Chain Hackathon.
\item We are layout a first Test Wafer for technology parameter measurement.
\item Currently re-viewed the Test Wafer and compress them now for more Chips per Wafer.
\end{itemize}
\end{frame}

\begin{frame}
\frametitle{Links:}
\begin{itemize}
\item Process
https://github.com/libresilicon/libresiliconprocess
\item Test Wafer
https://github.com/chipforge/PearlRiver
\item Standard Cell Library
https://github.com/chipforge/StdCellLib
\item Tool Chain
https://github.com/leviathanch/qtflow
\end{itemize}
\end{frame}


%------------------------------------------------
\begin{frame}
What still left
\end{frame}
%------------------------------------------------

\begin{frame}
\frametitle{To Do:}
\begin{itemize}
\item Shrink PearlRiver Test Wafer
\item Next Review before ordering the Masks
\item Documentation about what and how we like to measure Parameters
\item Transfer Parameters into Spice BSIM3v3 models
\item Manufacture a couple of Wafers and doing Measurement at HKUST
\item Process refinement
\item Finish Standard Cells
\item Install process Foundry for mass production
\item Manufacture first Microcontroller Chip in 2019
\end{itemize}
\end{frame}

%------------------------------------------------
\begin{frame}
Targets
\end{frame}
%------------------------------------------------

\begin{frame}
\frametitle{License:}
\begin{itemize}
\item Free and Open Source - while real Hardware GPL or BSD does not work.
\item Others like CERN we already evaluated.
\item We like that everybody can use the Process (even in your Basement),
\item including Universities and real foundries.
\end{itemize}
\end{frame}

\begin{frame}
\frametitle{Transfer-able}
\begin{itemize}
\item Everybody should have the possibility to transfer own designs into other foundries.
\item Foundries can compete in production cost and / or corporate.
\item Usable for Education also, while even analog designs heavy depends on process parameters.
\end{itemize}
\end{frame}

%------------------------------------------------
\begin{frame}
Contacts
\end{frame}
%------------------------------------------------

\begin{frame}
\frametitle{Mumble:}
\begin{itemize}
\item Every Sunday 21 p.m Hong Kong Time
\item Server 109.109.202.102, Port 64738
\end{itemize}
\end{frame}

%------------------------------------------------

\begin{frame}
\frametitle{Mailing List:}
\begin{itemize}
\item https://list.o2s.ch/mailman/listinfo/libre-silicon-devel
\end{itemize}
\end{frame}

%------------------------------------------------

\begin{frame}{Thanks!}
	\begin{center}
		\textbf{Dziekuje!} \\
		\textbf{Thank you very much!} \\
	\end{center}
\end{frame}

%------------------------------------------------

\begin{frame}
\frametitle{once again:}
\begin{itemize}
\item Mailing List
https://list.o2s.ch/mailman/listinfo/libre-silicon-devel
\item Process
https://github.com/libresilicon/libresiliconprocess
\item Test Wafer
https://github.com/chipforge/PearlRiver
\item Standard Cell Library
https://github.com/chipforge/StdCellLib
\item Layout Software
https://github.com/leviathanch/qtflow
\end{itemize}
\Huge{\centerline{You can help :-)}}
\end{frame}

%------------------------------------------------

\begin{frame}
\Huge{\centerline{The End}}
\end{frame}

%----------------------------------------------------------------------------------------

\end{document}
